\documentclass{beamer}

\usepackage[utf8]{inputenc}
\usepackage[ngerman]{babel}
\usepackage[T1]{fontenc}

\usetheme{Berlin}

\author{Ich Ichson}
\title{Ein Beispiel}
\subtitle{Ein Untertitel}
\begin{document}

\begin{frame}
\maketitle
\end{frame}

\begin{frame}
\tableofcontents
\end{frame}

\section{Einführung}

\subsection{Ein Unterabschnitt}

\begin{frame}{Folienname}{Beispiel}
Hallo

\begin{itemize}
\item Beispiel
\item Beispiel 2
\item Beispiel 3
\end{itemize}


Themes auf:
\url{http://deic.uab.es/~iblanes/beamer_gallery/}

\end{frame}

\begin{frame}{Ein Titel}

Eine inhaltsleere Folie

\end{frame}

\begin{frame}{Ein anderer Titel}

Leider kein Inhalt.

\end{frame}

\section{Beispiele}

\begin{frame}{Titel}

 Spaß mit Overlays!\dots
  \begin{itemize}
  \item Mit Hilfe des \texttt{pause}-Kommandos:
    \begin{itemize}
    \item Zuerst.
      \pause 
    \item Zweites.
    \end{itemize}
  \item
    Das Overlay spezifizieren:
    \begin{itemize}
    \item<3->
      Ich bin beim 3. Klick da.
    \item<4-5>
      Als 4 bis zum 5., dann nicht mehr.
    \end{itemize}
  \item
    Das \texttt{uncover}-Kommando:
    \begin{itemize}
      \uncover<5->{\item
        Klick 5.}
      \uncover<6->{\item
        Klick Nummer 6.}
    \end{itemize}
  \end{itemize}

\end{frame}

\end{document}